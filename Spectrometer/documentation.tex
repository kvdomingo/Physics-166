\documentclass[12pt,a4paper,twoside]{article}
\input{166.dat}
\usepackage{gensymb}
\usepackage{amsthm}
\usepackage{float}
\usepackage{siunitx}
\usepackage{amssymb}
\usepackage{float}
\usepackage{enumerate}
\usepackage{listings}
\usepackage{mathtools}
\usepackage[none]{hyphenat}
\usepackage{physics}
\newcommand\ddfrac[2]{\frac{\displaystyle #1}{\displaystyle #2}}
%\renewcommand{\familydefault}{\sfdefault}
\usepackage{booktabs,tabularx}
\renewcommand{\tabularxcolumn}{m}
\usepackage{listings}
\PassOptionsToPackage{hyphens}{url}\usepackage{hyperref}
\usepackage{color, colortbl}
\definecolor{cyan}{rgb}{0.85,0.89,0.95}
\renewcommand{\familydefault}{\sfdefault}

\begin{document}

\begin{titlepage}
\begin{center}
\vspace*{\fill}

\Huge{ Live-Feed-over-LAN Camera Spectrometer (LoLAN-CaS) Documentation} \\

\qquad
\qquad

\normalsize{Members: \\ 
Andrea Rica Advincula \\
Creo Baylon \\
Kenneth Domingo \\
Rene Principe Jr. \\
Lou Josef Tan}

\vspace*{\fill}
\end{center}
\end{titlepage}

\setcounter{page}{1}

\section{Overview}\label{sec:overview}
\medskip
The Live-Feed-over-LAN Camera Spectrometer (LoLAN-CaS) is an implementation of a spectrometer which can use any Android-based phone camera. On the hardware end, the camera broadcasts through a local area network (LAN) using a pre-set IP address. On the software end, the feed can be retrieved, processed, and displayed in real-time through any Python interpreter on a device connected on the same network. The spectrometer program depends on the following Python libraries:

\begin{itemize}

\item Numpy
\item Matplotlib
\item Scipy
\item OpenCV
\item Peakutils
\item URLlib

\end{itemize}

The current features are as follows:

\begin{itemize}

\item Calibration information can be set within the program itself.
\item Live feed of camera and corresponding intensity profile of a selected line scan region can be displayed in real-time on a computer with the required dependencies installed.
\item Scale of relative intensity profile can be set by the initial camera exposure settings but is always normalized.

\end{itemize}

\section{Theory}\label{sec:theory}\medskip

\section{Setup}\label{sec:setup}\medskip

\section{Program}\label{sec:program}\medskip

\subsection{\texttt{\footnotesize{Spectrometer}\normalsize{.\_\_init\_\_(calibrationLocation, calibrationWavelengths, lowerPix, upperPix, lowerBound, upperBound)}}}
Instantiates the \texttt{Spectrometer} object and takes the calibration arguments.

\begin{table}[H]
    \caption{Program initialization.}
    \begin{tabular}{>{\columncolor{cyan}}p{2in} p{4in}}
        \hline
        \textbf{Parameters} & \texttt{calibrationLocation : array\_like} \\
        &   Pixel locations of the peaks of the calibration image. \\ 
        & \texttt{calibrationWavelengths : array\_like} \\
        &   Corresponding wavelengths of \texttt{calibrationLocation}. \\
        & \texttt{lowerPix : int} \\
        &   Specifies pixel location of \texttt{lowerBound} (optional). \\
        & \texttt{upperPix : int} \\
        &   Specifies pixel location of \texttt{lowerBound} (optional). \\
        & \texttt{lowerBound : float} \\
        &	Specifies wavelength lower bound. \\
        & \texttt{upperBound : float} \\
        &	Specifies wavelength upper bound. \\ \hline
    \end{tabular}
    \label{tab:prog-init}
\end{table}

\subsection{\texttt{\footnotesize{Spectrometer}\normalsize{.plotCalibration()}}}

Plots the calibration curve and corresponding pixel-to-wavelength equation using linear regression.

\subsection{\texttt{\footnotesize{Spectrometer}\normalsize{.LineScan\_snapshot(image\_name, peaks, window\_length, polyorder)}}}

\begin{table}[H]
    \caption{\texttt{LineScan\_snapshot} arguments.}
    \begin{tabular}{>{\columncolor{cyan}}p{2in} p{4in}}
        \hline
        \textbf{Parameters} & \texttt{image\_name : str} \\
        &   File name of locally-stored image. \\ 
        & \texttt{peaks : bool} \\
        &   Sets whether peak points should be indicated on intensity profile. \\
        & \texttt{window\_length : int} \\
        &   Specifies window length of Savitsky-Golay filter. \\
        & \texttt{polyorder : int} \\
        &   Specifies polynomial order of Savitsky-Golay filter. \\ \hline
    \end{tabular}
    \label{tab:prog-lssnapshot}
\end{table}

\subsection{\texttt{\footnotesize{Spectrometer}\normalsize{.LineScan\_live(URL, show\_peaks, window\_length, polyorder)}}}

\begin{table}[H]
    \caption{\texttt{LineScan\_live} arguments.}
    \begin{tabular}{>{\columncolor{cyan}}p{2in} p{4in}}
        \hline
        \textbf{Parameters} & \texttt{URL : str} \\
        &   IP address of capturing device (Android-based phone camera only). \\ 
        & \texttt{show\_peaks : bool} \\
        &   Sets whether peak points should be indicated on intensity profile. \\
        & \texttt{window\_length : int} \\
        &   Specifies window length of Savitsky-Golay filter. \\
        & \texttt{polyorder : int} \\
        &   Specifies polynomial order of Savitsky-Golay filter. \\ \hline
    \end{tabular}
    \label{tab:prog-lslive}
\end{table}

\section{Demonstration}



\section*{Appendix}
Source code: \\
\url{https://colab.research.google.com/drive/1VMUdZ9GGeLgUW5F7rmk0VZNeu9xxkdcU}.

%\bibliographystyle{spp-bst}
%\bibliography{bibfile}

%\raggedbottom

%\pagebreak
%\pagebreak[3]
\iffalse
\newpage

\renewcommand\thefigure{A\arabic{figure}} 
\setcounter{figure}{0}
\fi %removing this will break the code for some reason

\end{document}

